\supersection{Заключение}
\label{sec:conclusion}

В ходе выполнения данной выпускной работы была разработана система токенизации дипломов на базе блокчейн-технологий, направленная на решение актуальных проблем, связанных с подделкой, утратой и сложностью проверки подлинности образовательных документов. Внедрение блокчейн-технологий и смарт-контрактов позволило создать децентрализованную, надежную и прозрачную систему, которая обеспечивает высокую степень защиты данных и упрощает процесс верификации дипломов.

Выделение и реализация функциональных и нефункциональных требований обеспечили стабильную и безопасную работу системы. Проектирование архитектуры и клиентских приложений позволило пользователям легко взаимодействовать с системой и управлять своими образовательными достижениями.

Разработанная система обладает рядом существенных преимуществ, таких как повышенная безопасность, надежность и удобство использования. Эти качества делают систему привлекательной для внедрения в образовательные учреждения, способствуя повышению доверия со стороны работодателей и улучшению карьерных перспектив выпускников. Кроме того, система обеспечивает глобальную доступность и прозрачность данных.

Важно отметить, что данная работа демонстрирует эффективность и перспективность использования блокчейн-технологий в сфере образования. Внедрение таких систем может значительно улучшить управление образовательными данными, обеспечивая высокую степень защиты, доверия и прозрачности, что является ключевым фактором в условиях цифровой трансформации общества.