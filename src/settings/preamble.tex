% !TeX program  = xelatex
% !TeX encoding = UTF-8
% !TeX root     = diploma.tex

% Пакеты для работы с гиперссылками и метаданными PDF
\usepackage{hyperref}
\urlstyle{same}

%\usepackage{showframe}

\hypersetup{
    pdftoolbar=true,        % Показывать панель инструментов Acrobat
    pdfmenubar=true,        % Показывать меню Acrobat
    pdffitwindow=false,     % Подгонять окно под размер страницы при открытии
    pdfstartview={FitH},    % Устанавливать масштаб по ширине окна
    pdftitle={Выпускная квалификационная работа}, % Заголовок PDF-документа
    pdfauthor={М. Д. Панфилов}, % Автор документа
    pdfsubject={Разработка системы токенизации дипломов на основе блокчейн-технологий для цифровой верификации и децентрализованного хранения образовательных достижений}, % Тема документа
    pdfcreator={XeLaTeX},   % Создатель документа
    pdfproducer={XeLaTeX in VSCode}, % Программа-производитель
    pdfkeywords={final qualifying work} {nft diplomas} {qualification collectibles} {educational nft}, % Ключевые слова
    pdfnewwindow=true       % Открывать ссылки в новом окне
}

% Пакеты для работы с графикой и цветом
\usepackage{graphicx}
\usepackage{xcolor}

% Пакеты для работы с таблицами
\usepackage{array}
\usepackage{longtable}
\usepackage{multirow}
\usepackage{booktabs}
\usepackage{siunitx}

% Пакеты для форматирования текста
\usepackage{setspace}
\usepackage{caption}
\usepackage{microtype}
\usepackage{chngcntr}

% Пакеты для удобной работы с комментариями
\usepackage{comment}

% Пакет для вставки страниц из PDF-документов
\usepackage{pdfpages}

\usepackage{etoolbox}

% Пакет для поддержки умных кросс-ссылок
\usepackage{cleveref}

% Пакет для управления плавающими объектами
\usepackage{float}

%\usepackage[compress]{cite}

% Настройки переносов и гипенизации
\tolerance=5000
\hbadness=9999
\emergencystretch=10pt
\hyphenpenalty=5000
\exhyphenpenalty=100

% Определение новых типов столбцов для таблиц
\newcolumntype{x}[1]{>{\setstretch{0.8}\small\centering\arraybackslash\hspace{0pt}}m{#1}}
\newcolumntype{y}[1]{>{\centering\arraybackslash\hspace{0pt}}m{#1}}

% Определение подписей и гиперссылок
\captionsetup[lstlisting]{justification=raggedright, singlelinecheck=false}

% Настройка отступа после подписи таблицы
\captionsetup[figure]{belowskip=-16pt}

\crefformat{table}{#2#1#3}
\crefrangeformat{table}{#3#1#4--#5#2#6}
\crefmultiformat{table}{#2#1#3}{ и~#2#1#3}{, #2#1#3}{ и~#2#1#3}

\definecolor{delim}{RGB}{20,105,176}
\definecolor{numb}{RGB}{106, 109, 32}
\definecolor{string}{rgb}{0.64,0.08,0.08}

\usepackage{listings}
% \usepackage{listingsutf8} % Для поддержки UTF-8 в листингах

\makeatletter
\renewenvironment{thebibliography}[1]
     {\section*{\refname}%
      \@mkboth{\MakeUppercase\refname}{\MakeUppercase\refname}%
      \list{\@biblabel{\@arabic\c@enumiv}}%
           {\settowidth\labelwidth{\@biblabel{#1}}%
            \leftmargin\labelwidth
            \advance\leftmargin\labelsep
            \setlength{\itemsep}{0pt} % Устанавливаем межстрочный интервал между элементами библиографии
            \setlength{\parsep}{0pt} % Устанавливаем интервал перед абзацами
            \setlength{\parskip}{0pt} % Устанавливаем интервал между абзацами
            \@openbib@code
            \usecounter{enumiv}%
            \let\p@enumiv\@empty
            \renewcommand\theenumiv{\@arabic\c@enumiv}}%
      \sloppy
      \clubpenalty4000
      \@clubpenalty \clubpenalty
      \widowpenalty4000%
      \sfcode`\.\@m}
     {\def\@noitemerr
       {\@latex@warning{Empty `thebibliography' environment}}%
      \endlist}
\makeatother

\lstset{
  extendedchars=true,       % Расширенные символы
  basicstyle=\monofont\footnotesize, % Шрифт и размер текста
  numbers=none,             % Расположение номеров строк (left, right)
  numberstyle=\color{gray}, % Стиль номеров строк
  backgroundcolor=\color{white}, % Цвет фона
  showspaces=false,         % Показывать пробелы
  showstringspaces=true,   % Показывать пробелы в строках
  frame=single,            % Рамка вокруг листинга
  showtabs=true,           % Показывать табуляцию
  frame=single,           % Рамка вокруг кода
  rulecolor=\color{black},  % Цвет рамки
  tabsize=2,                % Размер табуляции
  captionpos=t,             % Позиция заголовка (t или b)
  breaklines=true,          % Перенос длинных строк
  breakatwhitespace=false,  % Перенос строк по пробелам
  title=\lstname,           % Показывать название файла
  keywordstyle=\color{violet}, % Стиль ключевых слов
  commentstyle=\color{teal},   % Стиль комментариев
  stringstyle=\color{red},     % Стиль строк
  escapeinside={\%*}{*)},      % Для вставки LaTeX в комментариях
  morekeywords={*,...}         % Если нужно добавить ключевые слова
}

% Определение языка JavaScript для листингов
\lstdefinelanguage{JavaScript}{
  keywords={typeof, new, true, false, catch, function, return, null, switch, var, if, in, while, do, else, case, break},
  keywordstyle=\color{blue}\bfseries,
  ndkeywords={class, export, boolean, throw, implements, import, this},
  ndkeywordstyle=\color{darkgray}\bfseries,
  identifierstyle=\color{black},
  sensitive=false,
  comment=[l]{//},
  morecomment=[s]{/*}{*/},
  commentstyle=\color{purple}\ttfamily,
  stringstyle=\color{red}\ttfamily,
  morestring=[b]',
  morestring=[b]"
}
% Определение языка Solidity для листингов
\lstdefinelanguage{Solidity}{
  keywords={pragma, solidity, contract, function, returns, return, if, else, for, while, do, break, continue, throw, revert, emit, mapping, struct, enum, address, bool, string, int, uint, public, private, internal, external, pure, view, constant, payable, override, virtual, new, delete, require, assert, constructor, true, false, this},
  keywordstyle=\color{blue}\bfseries,
  ndkeywords={import, using, as, library, is, modifier, event, abstract, interface},
  ndkeywordstyle=\color{darkgray}\bfseries,
  identifierstyle=\color{black},
  sensitive=true,
  comment=[l]{//},
  morecomment=[s]{/*}{*/},
  commentstyle=\color{purple}\ttfamily,
  stringstyle=\color{red}\ttfamily,
  morestring=[b]',
  morestring=[b]"
}
% Определение языка Python для листингов
\lstdefinelanguage{Python}{
    keywords={def, return, raise, from, import, as, class, if, elif, else, while, continue, break, try, except, finally, with, assert, yield, lambda, global, nonlocal, pass, del, or, and, not, is, in, True, False, None},
    keywordstyle=\color{blue}\bfseries,
    ndkeywords={self, cls, __init__, __name__, __main__},
    ndkeywordstyle=\color{teal}\bfseries,
    identifierstyle=\color{black},
    sensitive=true,
    comment=[l]{\#},
    morecomment=[s]{"""}{"""},
    morecomment=[s]{'''}{'''},
    commentstyle=\color{purple}\ttfamily,
    stringstyle=\color{red}\ttfamily,
    morestring=[b]',
    morestring=[b]"
}
% Определение языка React для листингов
\lstdefinelanguage{React}{
    keywords={typeof, new, true, false, catch, function, return, null, switch, var, if, in, while, do, else, case, break, const, let, of},
    keywordstyle=\color{blue}\bfseries,
    ndkeywords={class, export, boolean, throw, implements, import, this, extends, component, render, constructor, super, props, state, setState, useState, useEffect, useContext, useReducer, useCallback, useMemo, useRef, Fragment},
    ndkeywordstyle=\color{teal}\bfseries,
    identifierstyle=\color{black},
    sensitive=false,
    comment=[l]{//},
    morecomment=[s]{/*}{*/},
    commentstyle=\color{purple}\ttfamily,
    stringstyle=\color{red}\ttfamily,
    morestring=[b]',
    morestring=[b]",
    morestring=[s]{`}{`},
    moredelim=[s][\color{orange}]{<}{>},
    moredelim=[s][\color{orange}]{</}{>},
    moredelim=[l][\color{orange}]{/>},
    moredelim=[l][\color{orange}]{=},
    moredelim=[s][\color{cyan}]{\{}{\}},
}

% Создание введения или заключения
\newcommand{\supersection}[1]{
	\section*{#1}
	\phantomsection
	\addcontentsline{toc}{section}{#1}
}