% !TeX program  = xelatex
% !TeX encoding = UTF-8

\usepackage{hyperref}
\hypersetup{pdftitle={ВКР на тему "<Разработка системы токенизации дипломов на основе блокчейн-технологий для цифровой верификации и децентрализованного хранения образовательных достижений">}, pdfauthor={М. Д. Панфилов}}

\usepackage{comment}

\usepackage{graphicx}

\usepackage{listings}
\usepackage{xcolor}
\lstset{basicstyle=\footnotesize, breaklines=true, numbers=left, captionpos=t, showstringspaces=false, commentstyle=\color{teal}, stringstyle=\color{red}, keywordstyle=\color{violet}}  % Настройки, применяемые ко всем листингам

% Создание введения или заключения
\newcommand{\supersection}[1]{
	\section*{#1}
	\phantomsection
	\addcontentsline{toc}{section}{#1}
}

\tolerance=5000
\hbadness=9999
\emergencystretch=10pt
\hyphenpenalty=5000
\exhyphenpenalty=100

\usepackage{caption}
\captionsetup[lstlisting]{justification=raggedright, singlelinecheck=false}

\usepackage{pdfpages}

\usepackage{microtype}

\usepackage{multirow}

\usepackage{array}
\usepackage{setspace}
\newcolumntype{x}[1]{>{\setstretch{0.8}\small\centering\arraybackslash\hspace{0pt}}m{#1}}
\newcolumntype{y}[1]{>{\centering\arraybackslash\hspace{0pt}}m{#1}}
\usepackage{longtable}

\usepackage{float}

\newcommand{\artauthor}[3]{
    \begin{FlushRight}
    \textbf{#1}\\ % Имя автора
    #2\\ % Статус/позиция
    \texttt{#3} % Email
    \end{FlushRight}
}
