\section{Специальный раздел}
\label{sec:special}

\subsection{Требования к разрабатываемой системе}

В данном разделе представлены ключевые аспекты разработки системы токенизации дипломов на основе блокчейн-технологий для цифровой верификации и децентрализованного хранения образовательных достижений. 

Требования к разрабатываемому программному продукту описывают набор параметров и характеристик, необходимых для полноценного функционирования системы. Они определяют функциональность системы, а также поведение и условия, необходимые для стабильной работы. Такие требования классифицируются на функциональные и нефункциональные.

\hyperref[subsec:func_req]{Функциональные требования} определяют конкретные задачи и операции, которые должна выполнять система для достижения заранее заданных целей. Описывается поведение в различных сценариях, включая процессы обработки данных, взаимодействие с пользователями, а также выполнение основных и дополнительных функций. Функциональные требования являются измеримыми и точно определенными; их выполнение можно однозначно проверить с помощью тестирования. Они играют важную роль в разработке систем, так как непосредственно влияют на архитектуру и проектные решения.

\hyperref[subsec:nonfunc_req]{Нефункциональные требования} формулируют качественные атрибуты системы, которые не влияют непосредственно на конкретные действия, но определяют общее качество и условия работы. Они включают производительность, безопасность, доступность, удобство использования, совместимость, масштабируемость, а также методики к тестированию и документации. Эти требования устанавливают ограничения и стандарты, которым должен соответствовать программный продукт, чтобы удовлетворять ожидания пользователей и обеспечивать надёжную и стабильную работу. Нефункциональные требования критически важны для обеспечения высокого уровня пользовательского опыта и управляемости системы на протяжении всего срока эксплуатации.

Функциональные и нефункциональные требования занимают важное место в процессе разработки продукта. Они формируют фундамент для проектирования, реализации и последующего тестирования системы. Без точно сформулированных задач создание эффективного и надежного программного продукта, отвечающего потребностям пользователей и бизнес-задачам, становится невозможным.

\subsubsection{Функциональные требования}
\label{subsec:func_req}

В рамках подраздела рассматриваются функциональные требования к разрабатываемому программному продукту:

\begin{enumerate}
    \item Создание коллекции
    \begin{itemize}
        \item Структура коллекций
        \item Регистрационные возможности
        \item Уникальность и доступ
    \end{itemize}
    \item Создание дипломов
    \begin{itemize}
        \item Автоматизация выпуска
        \item Привязка данных
        \item Образовательные достижения
    \end{itemize}
    \item Сокрытие персональных данных
    \begin{itemize}
        \item Методы защиты данных
        \item Минимизация данных
    \end{itemize}
    \item Мультиподпись
    \item Верификация диплома
    \item Интеграция с Telegram
    \item Поддержка кастодиальных и некастодиальных кошельков
    \item Обработка запросов через собственный RPC-узел
\end{enumerate}

\subsubsection{Нефункциональные требования}
\label{subsec:nonfunc_req}

Для обеспечения полноценной работоспособности и надежности разрабатываемой системы токенизации дипломов на базе блокчейн-технологий для цифровой верификации и децентрализованного хранения образовательных достижений, особое внимание уделяется формулировке нефункциональных требований. Они задают качественные стандарты для архитектуры системы, определяя общую производительность, безопасность, доступность и удобство использования. Наиболее важными из них являются:

\begin{itemize}
    \item \textit{Безопасность и конфиденциальность:} cистема должна использовать передовые методы криптографической защиты. Механизмы мультиподписи должны быть реализованы так, чтобы обеспечить необходимый уровень доверия и верификации среди всех участвующих сторон.
    \item \textit{Высокая доступность:} система должна обеспечивать стабильную и непрерывную работу в режиме 24/7. Это критически важно для пользователей, которые могут нуждаться в проверке подлинности достижений в любое время из любой точки мира.
    \item \textit{Производительность:} система должна обеспечивать быстрое время ответа и высокую пропускную способность при обработке транзакций, что особенно важно в периоды высоких нагрузок, например, во время выпускных экзаменов и вручения дипломов.
    \item \textit{Интероперабельность:} система должна быть совместима с различными блокчейнами, основанными на EVM (Ethereum Virtual Machine), особенно с сетями Polygon и Siberium, выбранными в аналитическом разделе. Важно предусмотреть возможность интеграции с другими образовательными и информационными системами через стандартизированные двоичные интерфейсы приложения (Application Binary Interface, ABI).
    \item \textit{Удобство использования:} клиентский интерфейс должен быть простым и интуитивно понятным, чтобы пользователи всех уровней могли легко управлять своими дипломами и проводить верификацию без дополнительных сложностей.
\end{itemize}

Реализация этих нефункциональных требований позволит создать систему, которая не только технически продвинута, но и удобна в использовании, безопасна и надежна, что обеспечит долгосрочную устойчивость системы и её способность адаптироваться к изменяющимся технологическим и операционным условиям.