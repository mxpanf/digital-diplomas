\begin{thebibliography}{99\kern\bibindent}
	% Section main (1)
	\begin{comment}
	\bibitem{bib:vk_nft_diploma} Пользователи ВКонтакте смогут подтвердить своё образование с помощью NFT-дипломов // VK Press URL: \url{https://vk.com/press/nft-diploma} (дата обращения: 04.03.2024).

	\bibitem{bib:mipt_nft_diploma} Выпускники магистратуры «Блокчейн» получили дипломы в виде NFT // Магистратура МФТИ по технологиям блокчейна URL: \url{https://blockchain.mipt.ru/news/40} (дата обращения: 04.03.2024).

	% Section special (2)
	\bibitem{bib:func_and_nf_req} Самое полное руководство по управлению требованиями и отслеживаемости // Visure Solutions URL: \url{https://visuresolutions.com/ru/requirements-management-traceability-guide/functional-vs-non-functional-requirements} (дата обращения: 07.04.2024).

	\bibitem{bib:app_req} Определение требований к программному обеспечению  // Narod.ru URL: \url{https://chaox-dump.narod.ru/ooap/ooapl6.html} (дата обращения: 07.04.2024).

	\bibitem{bib:152fz} Федеральный закон от 27.07.2006 г. № 152-ФЗ // Kremlin.ru URL: \url{http://kremlin.ru/acts/bank/24154} (дата обращения: 07.04.2024).
	
	\bibitem{bib:sc_abi} Финогеев А. Г., Васин С. М., Гамидуллаева Л. А., Финогеев А. А. // Технология смарт контрактов на основе блокчейн для минимизации трансакционных издержек в региональных инновационных системах // Вопросы безопасности. 2018. №3: 46-48. % https://cyberleninka.ru/article/n/tehnologiya-smart-kontraktov-na-osnove-blokcheyn-dlya-minimizatsii-transaktsionnyh-izderzhek-v-regionalnyh-innovatsionnyh-sistemah

	\bibitem{bib:ipfs_is} IPFS вместо HTTP — будущее децентрализованного интернета // ForkLog URL: \url{https://forklog.com/cryptorium/chto-takoe-ipfs} (дата обращения: 07.04.2024).
	\end{comment}
	
	% Section tech (3)
	\bibitem{bib:ide_is} Что такое интегрированная среда разработки (IDE)? // Amazon Web Services URL: \url{https://aws.amazon.com/ru/what-is/ide} (дата обращения: 19.05.2024).

	\bibitem{bib:webstorm} WebStorm IDE // JetBrains URL: \url{https://www.jetbrains.com/ru-ru/webstorm} (дата обращения: 19.05.2024).

	\bibitem{bib:vscode} Visual Studio Code is a distribution of the Code // GitHub URL: \url{https://github.com/microsoft/vscode} (дата обращения: 19.05.2024).

	\bibitem{bib:neovim} Neovim — текстовый редактор с TUI интерфейсом // ALT Gnome URL: \url{https://alt-gnome.wiki/neovim.html} (дата обращения: 19.05.2024).

	\bibitem{bib:fastapi} FastAPI is a modern, fast (high-performance), web framework for building APIs // Tiangolo URL: \url{https://fastapi.tiangolo.com} (дата обращения: 19.05.2024).

	\bibitem{bib:evm} Ethereum Virtual Machine (EVM) // Ethereum Org URL: \url{https://ethereum.org/en/developers/docs/evm} (дата обращения: 19.05.2024).

	\bibitem{bib:react} The library for web and native user interfaces // React Dev URL: \url{https://react.dev} (дата обращения: 19.05.2024).

	\bibitem{bib:caddy} API or ABI changing // Caddy URL: \url{https://caddyserver.com} (дата обращения: 19.05.2024).

	\bibitem{bib:erc721} ERC-721: Non-Fungible Token Standard // Ethereum Org URL: \url{https://eips.ethereum.org/EIPS/eip-721} (дата обращения: 19.05.2024).

	\bibitem{bib:abi_is} API or ABI changing // ALT Linux Wiki URL: \url{https://www.altlinux.org/API_or_ABI_changing} (дата обращения: 19.05.2024).

	\bibitem{bib:dapps} Decentralized Applications (dApps) // Investopedia URL: \url{https://www.investopedia.com/terms/d/decentralized-applications-dapps.asp} (дата обращения: 19.05.2024).

	

	% Section economy (4)
	\begin{comment}
	\bibitem{bib:reestrpo} Классификатор ПО (в ред. приказов Минцифры России от 22.09.2020 № 486, 26.04.2022 № 393, от 22.12.2022 № 974) // ЦКИТ URL: \url{https://ru-ikt.ru/reestrpo} (дата обращения: 04.05.2024).

	\bibitem{bib:reestrpo_docs_list} Перечень документов для внесения ПО в отечественный реестр // Гардиум URL: \url{https://legal-support.ru/information/blog/programmy/perechen-dokumentov-dlya-vneseniya-po-v-reestr} (дата обращения: 04.05.2024).

	\bibitem{bib:cocomoii_gen} Лекции по управлению программными проектами // CIT Forum URL: \url{http://citforum.ru/SE/project/arkhipenkov_lectures/13.shtml} (дата обращения: 04.05.2024).

	\bibitem{bib:cocomoii_win} Садовский И. Д. // Применение модели COCOMO II для оценки разработки программного обеспечения в Windows проектах // Экономика и бизнес: теория и практика. 2016. №10. % https://cyberleninka.ru/article/n/primenenie-modeli-cocomo-ii-dlya-otsenki-razrabotki-programmnogo-obespecheniya-v-windows-proektah

	\bibitem{bib:scale_f} Миньков С. Л. // Программная инженерия, Лабораторный практикум // ТУСУР. 2014. Часть 2. URL: \url{https://asu.tusur.ru/learning/bak230700/d36/b230700_d36_labs2.pdf} (дата обращения: 04.05.2024).

	\bibitem{bib:tco_gartner} Use Total Cost of Ownership to Optimize Costs and Increase Savings // Gartner URL: \url{https://www.gartner.com/en/documents/3847267} (дата обращения: 04.05.2024).
	\end{comment}
\end{thebibliography}