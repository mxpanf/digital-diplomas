\begin{thebibliography}{99\kern\bibindent}
	% Section special (2)
	\begin{comment}
	\bibitem{bib:func_and_nf_req} Самое полное руководство по управлению требованиями и отслеживаемости // Visure Solutions URL: \url{https://visuresolutions.com/ru/requirements-management-traceability-guide/functional-vs-non-functional-requirements} (дата обращения: 07.04.2024).

	\bibitem{bib:app_req} Определение требований к программному обеспечению  // Narod.ru URL: \url{https://chaox-dump.narod.ru/ooap/ooapl6.html} (дата обращения: 07.04.2024).

	\bibitem{bib:152fz} Федеральный закон от 27.07.2006 г. № 152-ФЗ // Kremlin.ru URL: \url{http://kremlin.ru/acts/bank/24154} (дата обращения: 07.04.2024).
	
	\bibitem{bib:sc_abi} Финогеев А. Г., Васин С. М., Гамидуллаева Л. А., Финогеев А. А. // Технология смарт контрактов на основе блокчейн для минимизации трансакционных издержек в региональных инновационных системах // Вопросы безопасности. 2018. №3: 46-48. % https://cyberleninka.ru/article/n/tehnologiya-smart-kontraktov-na-osnove-blokcheyn-dlya-minimizatsii-transaktsionnyh-izderzhek-v-regionalnyh-innovatsionnyh-sistemah

	\bibitem{bib:ipfs_is} IPFS вместо HTTP — будущее децентрализованного интернета // ForkLog URL: \url{https://forklog.com/cryptorium/chto-takoe-ipfs} (дата обращения: 07.04.2024).
	\end{comment}

	% Section economy (4)
	\bibitem{bib:reestrpo} Классификатор ПО (в ред. приказов Минцифры России от 22.09.2020 № 486, 26.04.2022 № 393, от 22.12.2022 № 974) // ЦКИТ URL: \url{https://ru-ikt.ru/reestrpo} (дата обращения: 04.05.2024).

	\bibitem{bib:cocomoii_gen} Лекции по управлению программными проектами // CIT Forum URL: \url{http://citforum.ru/SE/project/arkhipenkov_lectures/13.shtml} (дата обращения: 04.05.2024).

	\bibitem{bib:cocomoii_win} Садовский И. Д. // Применение модели COCOMO II для оценки разработки программного обеспечения в Windows проектах // Экономика и бизнес: теория и практика. 2016. №10. % https://cyberleninka.ru/article/n/primenenie-modeli-cocomo-ii-dlya-otsenki-razrabotki-programmnogo-obespecheniya-v-windows-proektah

	\bibitem{bib:reestrpo_docs} Перечень документов для внесения ПО в отечественный реестр // Гардиум URL: \url{https://legal-support.ru/information/blog/programmy/perechen-dokumentov-dlya-vneseniya-po-v-reestr} (дата обращения: 04.05.2024).

	\bibitem{bib:scale_f} Миньков С. Л. // Программная инженерия, Лабораторный практикум // ТУСУР. 2014. Часть 2. URL: \url{https://asu.tusur.ru/learning/bak230700/d36/b230700_d36_labs2.pdf} (дата обращения: 04.05.2024).

	\bibitem{bib:labor_f} Алиев Х. Р. // Комбинированная модель оценки трудоемкости разработки программного обеспечения // π-Economy. 2010. №3 (99). % https://cyberleninka.ru/article/n/kombinirovannaya-model-otsenki-trudoemkosti-razrabotki-programmnogo-obespe-cheniya

	\bibitem{bib:cost_dis} Подсчет себестоимости часа разработки программного обеспечения // Хабр URL: \url{https://habr.com/ru/articles/256537} (дата обращения: 04.05.2024).

	\bibitem{bib:tco_gartner} Use Total Cost of Ownership to Optimize Costs and Increase Savings // Gartner URL: \url{https://www.gartner.com/en/documents/3847267} (дата обращения: 04.05.2024).

	\bibitem{bib:iso_9126} ГОСТ Р ИСО/МЭК 9126-93 <<Информационная технология. Оценка программной продукции. Характеристики качества и руководства по их применению>> // URL: \url{https://files.stroyinf.ru/Data/189/18984.pdf} (дата обращения: 04.05.2024).

\end{thebibliography}